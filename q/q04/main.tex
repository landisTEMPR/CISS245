%-*-latex-*-
%-*-latex-*-
\input{myquizpreamble}
\input{yliow}
\input{ciss245}
\textwidth=6in

\renewcommand\TITLE{Final f01}

\renewcommand\EMAIL{}
\newcommand\POINTS{\textwhite{xxx/xxx}}


%------------------------------------------------------------------------------
\nextq
Either write down the output of this code fragment or write ERROR
if it is not valid C\texttt{++}.
\begin{Verbatim}[fontsize=\scriptsize, frame=single]
int x = 0, y = 42, z = 99;
if (x < 0)
{
    int y = 1, z = 100;
    if (x == 0)
    {
        int y = 2;
        if (y < 1)
        {
            std::cout << z;
        }
        else
        {
            std::cout << z;
        }
    }
    else
    {
        int y = 3;
        if (y < 2)
        {
            std::cout << z;
        }
        else
        {
            std::cout << z;
        }
    }
}
else
{
    int y = 4, z = 200;
    if (x == 0)
    {
        if (y < 3)
        {
            std::cout << z;
        }
        else
        {
            std::cout << z;
        }
    }
    else
    {
        if (y < 4)
        {
            std::cout << z;
        }
        else
        {
            std::cout << z;
        }
    }
}
\end{Verbatim}
\vspace{-6pt}
\ANSWER
\begin{answercode}

\end{answercode}

%------------------------------------------------------------------------------
\nextq
Rewrite the following function so that the if-else statement is replaced by
the ternary operator. The function should contain only one statement.
\begin{Verbatim}[frame=single, fontsize=\footnotesize]
int sign(double x)
{
    int ret;
    if (x >= 0)
    {
        ret = 1;
    }
    else
    {
        ret -1;
    }
    return ret;
}
\end{Verbatim}
\vspace{-6pt}
\ANSWER
\begin{answercode}

\end{answercode}

%------------------------------------------------------------------------------
\nextq
Write a function
\begin{console}
int numdigits(int n);
\end{console}
that returns the number of digits in \verb!n!.
For instance if \verb!n! is 0 or 5 or 9, the function returns \verb!1!.
If \verb!n! is 23 or 42 or 99, the function returns \verb!2!.
If \verb!n! is -123 or -243 or -798, the function returns \verb!3!.
Etc.
\\
\ANSWER
\begin{answercode}

\end{answercode}

%------------------------------------------------------------------------------
\nextq
Write a function
\begin{console}
bool isprime(int n);
\end{console}
that returns \verb!true! if and only if \verb!n! is a prime.
For instance if \verb!n! is 2, 3, 5, 7, 11, 13, 17, 19, 23, or 29,
the function returns \verb!true!.
If \verb!n! is 0, 1, 4, 6, 8, 9, 10, 12, 14, or 15,
the function returns \verb!false!.
\\
\ANSWER
\begin{answercode}

\end{answercode}

%------------------------------------------------------------------------------
\input{thispostamble}