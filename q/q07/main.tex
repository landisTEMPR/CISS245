%-*-latex-*-
%-*-latex-*-
\input{myquizpreamble}
\input{yliow}
\input{ciss245}
\textwidth=6in

\renewcommand\TITLE{Final f01}

\renewcommand\EMAIL{}
\newcommand\POINTS{\textwhite{xxx/xxx}}


%------------------------------------------------------------------------------
\nextq
Write function
\verb!void remove(char s[], char c)!
such that it removes \verb!c! from C-string \verb!s!.
For instance if \verb!s! is \verb!"hello world"!,
then on calling \verb!remove(s, 'o')!,
\verb!s! becomes \verb!"hell wrld"!.
\\
\ANSWER
\begin{answercode}
void remove(char s[], char c)
{
}
\end{answercode}
(Hint on next page if needed.)

\newpage
\textsc{Hint}

All the information you need is in the chapter on C-strings.
The main thing being a C-string has a sentinel value to terminate
the data (i.e., characters) in the string.
It's the null character \verb!'\0'!.
You need to loop over the characters over \verb!s!
and copy it back to itself if the character you have read is not
the value of \verb!c!.
This means you need two indexing variables,
\verb!i! where you read a character \verb!s[i]!
and \verb!j! where you write the character.
Once you have copies the \verb!'\0'!, you stop since that's the last
thing to copy.

The the idea is therefore something like this:
\begin{console}[frame=single,fontsize]
let i = 0 and j = 0
while (1)
    if character of s at index i is not c:
        s[j] = s[i]
        ++j
    if character of s at index i is '\0':
        break
\end{console}
Note: The above pseudocode is not quite right.
You'll need to think about it more.

%------------------------------------------------------------------------------
\input{thispostamble}