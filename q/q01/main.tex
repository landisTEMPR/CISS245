%-*-latex-*-
%-*-latex-*-
\input{myquizpreamble}
\input{yliow}
\input{ciss245}
\textwidth=6in

\renewcommand\TITLE{Final f01}

\renewcommand\EMAIL{}
\newcommand\POINTS{\textwhite{xxx/xxx}}


This is a closed-book, no compiler, 5 minute quiz.

%------------------------------------------------------------------------------
\nextq
What is the output of the following code fragment?
\begin{console}[fontsize=\footnotesize]
int x = 42, y = 3;
std::cout << x / y << '\n';
\end{console}
\ANSWER
\begin{answercode}
14
\end{answercode}

%------------------------------------------------------------------------------
\nextq
What is the output of the following code fragment?
\begin{console}[fontsize=\footnotesize]
int x = 42, y = 2;
std::cout << x / y << '\n';
\end{console}
\ANSWER
\begin{answercode}
21
\end{answercode}

%------------------------------------------------------------------------------
\nextq
What is the output of the following code fragment?
\begin{console}[fontsize=\footnotesize]
int x = 42, y = 1;
std::cout << x / y << '\n';
\end{console}
\ANSWER
\begin{answercode}
42
\end{answercode}

%------------------------------------------------------------------------------
\nextq
What is the output of the following code fragment?
\begin{console}[fontsize=\footnotesize]
int x = 42, y = 0;
std::cout << x / y << '\n';
\end{console}
\ANSWER
\begin{answercode}

\end{answercode}

%------------------------------------------------------------------------------
\nextq
What is the output of the following code fragment?
\begin{console}[fontsize=\footnotesize]
int x = 21, y = 42;
std::cout << x / y << '\n';
\end{console}
\ANSWER
\begin{answercode}

\end{answercode}

%------------------------------------------------------------------------------
\nextq
What is the output of the following code fragment?
\begin{console}[fontsize=\footnotesize]
std::cout << 42 % 3 << '\n';
\end{console}
\ANSWER
\begin{answercode}

\end{answercode}

%------------------------------------------------------------------------------
\nextq
What is the output of the following code fragment?
\begin{console}[fontsize=\footnotesize]
std::cout << 42 % 2 << '\n';
\end{console}
\ANSWER
\begin{answercode}

\end{answercode}

%------------------------------------------------------------------------------
\nextq
What is the output of the following code fragment?
\begin{console}[fontsize=\footnotesize]
std::cout << 42 % 1 << '\n';
\end{console}
\ANSWER
\begin{answercode}

\end{answercode}

%------------------------------------------------------------------------------
\nextq
What is the output of the following code fragment?
\begin{console}[fontsize=\footnotesize]
std::cout << 42 % 0 << '\n';
\end{console}
\ANSWER
\begin{answercode}

\end{answercode}

%------------------------------------------------------------------------------
\nextq
What is the output of the following code fragment?
\begin{console}[fontsize=\footnotesize]
std::cout << 42 % 42 << '\n';
\end{console}
\ANSWER
\begin{answercode}

\end{answercode}

%------------------------------------------------------------------------------
\nextq
What is the string length of \verb!"hello \tworld\n???\n"!?
\\
\ANSWER
\begin{answercode}

\end{answercode}

%------------------------------------------------------------------------------
\nextq
In the following code fragment, the output is \verb!132!.
The value of integer variable \verb!n! is a power of 10.
What is the value of \verb!n!?
\begin{console}[fontsize=\footnotesize]
std::cout << 132435 / n << '\n';
\end{console}
\ANSWER
\begin{answercode}

\end{answercode}

%------------------------------------------------------------------------------
\nextq
In the following code fragment, the output is \verb!2435!.
The value of integer variable \verb!n! is a power of 10.
What is the value of \verb!n!?
\begin{console}[fontsize=\footnotesize]
std::cout << 132435 % n << '\n';
\end{console}
\ANSWER
\begin{answercode}

\end{answercode}

%------------------------------------------------------------------------------
\nextq
In the following code fragment, the output is \verb!32!.
The values of integer variables \verb!m! and \verb!n! are powers of 10.
What is the of \verb!m!?
\begin{console}[fontsize=\footnotesize]
std::cout << 132435 / m % n << '\n';
\end{console}
\ANSWER
\begin{answercode}

\end{answercode}

%------------------------------------------------------------------------------
\nextq
In the following code fragment, the output is \verb!32!.
The values of integer variables \verb!m! and \verb!n! are powers of 10.
What is the value of \verb!n!?
\begin{console}[fontsize=\footnotesize]
std::cout << 132435 / m % n << '\n';
\end{console}
\ANSWER
\begin{answercode}

\end{answercode}

%------------------------------------------------------------------------------
\nextq
\tf:
(T = true, F = false, M = statement is meaningless and cannot be answered.)
You cannot assign an integer value to a \verb!double! variable because
of type mismatch.
In other words the code fragment below is invalid \cpp.
\begin{console}[fontsize=\footnotesize]
double x = 42;
\end{console}
\ANSWER
\begin{answercode}

\end{answercode}

%------------------------------------------------------------------------------
\nextq
\tf:
(T = true, F = false, M = statement is meaningless and cannot be answered.)
You cannot assign an integer value to an integer array because
of type mismatch.
In other words the code fragment below is invalid \cpp.
\begin{console}[fontsize=\footnotesize]
int x[10] = 42;
\end{console}
\ANSWER
\begin{answercode}

\end{answercode}

%------------------------------------------------------------------------------
\nextq
What is the output of the following code fragment?
\begin{console}[fontsize=\footnotesize]
int x = 4, y = 8;
std::cout << double(x / y) << '\n';
\end{console}
\ANSWER
\begin{answercode}

\end{answercode}

%------------------------------------------------------------------------------
\nextq
If the output of the following code fragment is \verb!6!,
what is the smallest possible positive integer value for \verb!x!?
\begin{console}[fontsize=\footnotesize]
std::cout << 5 + 6 * x / 5 % 10 - 2 << '\n';
\end{console}
\ANSWER
\begin{answercode}

\end{answercode}

%------------------------------------------------------------------------------
\nextq
Complete the following code fragment that gets a 4-digit integer input from the
user and store it in \verb!x!,
assigns the digits of \verb!x! to
\verb!x0!, \verb!x1!, \verb!x2!, \verb!x3!
where \verb!x0! is the lowest order digit (the ones digit)
and \verb!x3! is the highest order digit (the thousands digit).
For instance if the user enters \verb!1462!, then the output of the
code fragment is \verb!1 4 6 2!.
\\
\ANSWER
\begin{answercode}
int x;
std::cin >> x;
int x0 = 0;
int x1 = 0;
int x2 = 0;
int x3 = 0;
std::cout << x3 << ' ' << x2 << ' ' << x1 << ' ' << x0 << '\n';
\end{answercode}

%------------------------------------------------------------------------------
\input{thispostamble}
