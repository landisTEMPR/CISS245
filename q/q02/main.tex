%-*-latex-*-
%-*-latex-*-
\newcommand\COURSE{ciss245}
\newcommand\ASSESSMENT{q2605}
\newcommand\ASSESSMENTTYPE{Quiz}
\newcommand\POINTS{\textwhite{xxx/xxx}}

\input{myquizpreamble}
\input{yliow}
\input{\COURSE}
%-*-latex-*-

%-*-latex-*-

%-*-latex-*-
\renewcommand\TITLE{\ASSESSMENTTYPE \ \ASSESSMENT}

\renewcommand\EMAIL{}
\renewcommand\AUTHOR{YOUR EMAIL}

\textwidth=6in
\begin{document}
\topmatterthree

%------------------------------------------------------------------------------
\nextq
Complete the following \cpp\ statement so that \verb!r! has a random integer
value in the interval $[5, 13]$,
i.e., the value is 5 or 6 or 7 or 8 or 9 or 10 or 11 or 12.
\\
\ANSWER
\begin{answercode}
int r =                    ;
\end{answercode}

%------------------------------------------------------------------------------
\nextq
Complete the following \cpp\ statement so that \verb!r! has a random integer value
in the interval $[-100, 100]$.
\\
\ANSWER
\begin{answercode}
int r =                    ;
\end{answercode}

%------------------------------------------------------------------------------
\nextq
Complete the following \cpp\ statement
so that \verb!r! has a random even integer value
in the interval $[-100, 100]$.
\\
\ANSWER
\begin{answercode}
int r =                    ;
\end{answercode}

%------------------------------------------------------------------------------
\nextq
Complete the following \cpp\ statement so that \verb!r! has a random double value
in the interval $[-1.1, 8.6]$.
\\
\ANSWER
\begin{answercode}
double r =                    ;
\end{answercode}

%------------------------------------------------------------------------------
\nextq
Write a \textit{complete program} that rolls two (random) dice until two
sixes are obtained.
Here's a test run
\begin{console}[fontsize=\footnotesize]
(1, 1) (2, 1) (6, 6)
\end{console}
\ANSWER
\begin{answercode}
#include <iostream>

int main()
{
}
\end{answercode}

%------------------------------------------------------------------------------
\nextq
Write a \textit{complete program} that rolls two (random) dice until two
sixes are obtained.
Furthemore, if the first roll is a 1, you do not roll the second die,
but you continue to the next try.
Here's a test run
\begin{console}[fontsize=\footnotesize]
(2, 3) (5, 3) (1) (4, 5) (6, 6)
\end{console}
Notice that in the third try, the first die is a 1, so you forget
about rolling the second die and go on to the fourth try.
\\
\ANSWER
\begin{answercode}
#include <iostream>

int main()
{
}
\end{answercode}

%------------------------------------------------------------------------------
%-*-latex-*-
\newpage
\input{instructions}
\end{document}
