%-*-latex-*-
%-*-latex-*-
\input{myquizpreamble}
\input{yliow}
\input{ciss245}
\textwidth=6in

\renewcommand\TITLE{Final f01}

\renewcommand\EMAIL{}
\newcommand\POINTS{\textwhite{xxx/xxx}}


%------------------------------------------------------------------------------
\nextq
Write down the output of the following print statement
or write ERROR if there is a program error.
\begin{console}
int i = 42;
int j = 5;
int k = -2;
int * p = &i;
int * q = &j;
int * r = p;
p = q;
q = r;
std::cout << *p << ' ' << *q << ' ' << *r << '\n';
\end{console}
\ANSWER
\begin{answercode}

\end{answercode}

%------------------------------------------------------------------------------
\nextq
Write down the output of the following print statement
or write ERROR if there is a program error.
\begin{console}
int i = 42;
int j = 5;
int k = -2;
int * p = &i;
int * q = &j;
int * r = &k;
std::cout << *p + *q + *r + *i << '\n';
\end{console}
\ANSWER
\begin{answercode}

\end{answercode}

%------------------------------------------------------------------------------
\nextq
Variables \verb!x! and \verb!p! are already declared.
Write one statement (semicolon please!) so that the memory model becomes:
\begin{python}
from latextool_basic import *
p = Plot()

p += Rect(x0=-1, y0=0.5, x1=0, y1=1.5, background='white', linewidth=0, label=r'\texttt{p}')

r0 = Rect(x0=0, y0=0.5, x1=1, y1=1.5, background='white', linewidth=0.05)
p += r0

p += Rect(x0=-1, y0=2, x1=0, y1=3, background='white', linewidth=0, label=r'\texttt{x}')
c = RectContainer(x=0, y=2)
for x in '51423':
    c += Rect2(x0=0, y0=0, x1=1, y1=1,
         linewidth=0.05,
         label=r'{\texttt{%s}}' % x)
p += c

# border
p += Rect(x0=-1.25, y0=-0, x1=6, y1=3.5, linewidth=0.1)

p += Line(points=[(r0.centerx(), r0.centery()),
                  (c[3].centerx(), r0.centery()),
                  (c[3].centerx(), c[3].bottomy())],
          startstyle='dot', linewidth=0.05,
          endstyle='->')
print(p)
\end{python}
\ANSWER
\begin{answercode}

\end{answercode}

%------------------------------------------------------------------------------
\nextq
At a specific point in time during the execution of
a program, the memory model of the variables look like this:
\begin{python}
from latextool_basic import *
p = Plot()

# p
p += Rect(x0=-1, y0=2, x1=0, y1=3, background='white', linewidth=0, label=r'\texttt{p}')
pbox = Rect(x0=0, y0=2, x1=1, y1=3, background='white', linewidth=0.05)
p += pbox

# q
p += Rect(x0=-1, y0=0.5, x1=0, y1=1.5, background='white', linewidth=0, label=r'\texttt{q}')
qbox = Rect(x0=0, y0=0.5, x1=1, y1=1.5, background='white', linewidth=0.05)
p += qbox

abox = Rect(x0=3, y0=2, x1=4, y1=3, background='white', linewidth=0.05,
            label=r'\texttt{0}')
p += abox

# border
p += Rect(x0=-1.25, y0=0, x1=5, y1=3.5, linewidth=0.1)

p += Line(points=[(pbox.centerx(), pbox.centery()),
                  (abox.leftx(), abox.centery())],
          startstyle='dot', linewidth=0.05,
          endstyle='->')

print(p)
\end{python}
Write one statement that will cause the memory model to become
\begin{python}
from latextool_basic import *
p = Plot()

# p
p += Rect(x0=-1, y0=2, x1=0, y1=3, background='white', linewidth=0, label=r'\texttt{p}')
pbox = Rect(x0=0, y0=2, x1=1, y1=3, background='white', linewidth=0.05)
p += pbox

# q
p += Rect(x0=-1, y0=0.5, x1=0, y1=1.5, background='white', linewidth=0, label=r'\texttt{q}')
qbox = Rect(x0=0, y0=0.5, x1=1, y1=1.5, background='white', linewidth=0.05)
p += qbox

abox = Rect(x0=3, y0=2, x1=4, y1=3, background='white', linewidth=0.05,
            label=r'\texttt{0}')
p += abox

# border
p += Rect(x0=-1.25, y0=0, x1=5, y1=3.5, linewidth=0.1)

p += Line(points=[(pbox.centerx(), pbox.centery()),
                  (abox.leftx(), abox.centery())],
          startstyle='dot', linewidth=0.05,
          endstyle='->')

p += Line(points=[(qbox.centerx(), qbox.centery()),
                  (abox.leftx(), abox.centery())],
          startstyle='dot', linewidth=0.05,
          endstyle='->')

print(p)
\end{python}
\ANSWER
\begin{answercode}

\end{answercode}

%------------------------------------------------------------------------------
\input{thispostamble}