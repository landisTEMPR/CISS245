%-*-latex-*-
%-*-latex-*-
\input{myquizpreamble}
\input{yliow}
\input{ciss245}
\textwidth=6in

\renewcommand\TITLE{Final f01}

\renewcommand\EMAIL{}
\newcommand\POINTS{\textwhite{xxx/xxx}}


%------------------------------------------------------------------------------
\nextq
In the answer box below, allocate memory for \verb!p! and \verb!q! to point to:
\\
(a) write one statement so that \verb!p! points to an array of size 10 in
the heap and
\\
(b) write one statement so that \verb!q! points to the third value
(i.e., index 2 value) of the array that \verb!p! points to.
\\
\ANSWER
\begin{answercode}
double *p, *q;

\end{answercode}

%------------------------------------------------------------------------------
\nextq
The function below has memory issues. Fix the problem.
\\
\ANSWER
\begin{answercode}
void f()
{
    double * p = new double;
    char * q = new char[1024];
    // Write two statements below to fix the memory issues

    return;
}
\end{answercode}

%------------------------------------------------------------------------------
\nextq
There is an array with values 2, 3, 5, 7, 11 in the heap.
A pointer \verb!p! is pointing to the value 7 of this array.
Complete the statement below so that \verb!q! points to the value 3
in the above array.
\\
\ANSWER
\begin{answercode}
int * q =          ;
\end{answercode}

%------------------------------------------------------------------------------
\nextq
Complete the following linear search function so that
if \verb!p! is returned, then \verb!p! points to the first time
\verb!target! appears in the array with beginning address
\verb!start! and ending address \verb!end - 1!.
If \verb!target! is not found, then \verb!NULL! is returned.
\\
\ANSWER
\begin{answercode}
int * linearsearch(int * start, int * end, int target)
{
    for (int * p =          ; p <          ; ++p)
    {
        if (              )
        {
            return            ;
        }
    }
    return              ;
}
\end{answercode}
(This should be a straightforward translation of the linear search
algorithm from CISS240. Do not change the algorithm other than
translating it from array indexing to pointer scanning.)

%------------------------------------------------------------------------------
\input{thispostamble}