%-*-latex-*-
%-*-latex-*-
\newcommand\COURSE{ciss245}
\newcommand\ASSESSMENT{q2605}
\newcommand\ASSESSMENTTYPE{Quiz}
\newcommand\POINTS{\textwhite{xxx/xxx}}

\input{myquizpreamble}
\input{yliow}
\input{\COURSE}
%-*-latex-*-

%-*-latex-*-

%-*-latex-*-
\renewcommand\TITLE{\ASSESSMENTTYPE \ \ASSESSMENT}

\renewcommand\EMAIL{}
\renewcommand\AUTHOR{YOUR EMAIL}

\textwidth=6in
\begin{document}
\topmatterthree

%------------------------------------------------------------------------------
\nextq
The following program does not run.
Insert \textit{one} statement to correct the problem.
\\
\ANSWER
\begin{answercode}
#include <iostream>

int main()
{
    std::cout << sum(5, 10, 1) << std::endl;
    return 0;
}

int sum(int start, int end, int step)
{
    int s = 0;
    for (int i = start; i <= end; i += step)
    {
        s += i;
    }
    return s;
}
\end{answercode}

%------------------------------------------------------------------------------
\nextq
The function \verb!play_audio()! plays music file an audio file. The parameters are:
\begin{enumerate}[nosep]
\li \verb!filename!: a C-string that is the name of the audio file to be load
\li \verb!track_number!: an integer.
$0$ is the first track, $1$ is the second track, etc.
$-1$ is \lq\lq play all tracks".
\li \verb!repeat!: a boolean.
\end{enumerate}
The function returns a $0$ if there are no errors,
a $-1$ if the file cannot be found using the \verb!filename!,
$-2$ if the \verb!track_number! is not $-1$ and the track number is not found
in the audio file.
Write down the prototype of this function.
\\
\ANSWER
\begin{answercode}

\end{answercode}

%------------------------------------------------------------------------------
\nextq
The following function call
\begin{console}
push_back(x, x_len, 42);
\end{console}
sets \verb!x[x_len]! to \verb!42! and increments \verb!x_len! by 1.
\verb!x! is an array of integer values.
Write down the function prototype of \verb!push_back!.
\\
\ANSWER
\begin{answercode}

\end{answercode}

%------------------------------------------------------------------------------
%-*-latex-*-
\newpage
\input{instructions}
\end{document}
