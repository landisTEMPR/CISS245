%-*-latex-*-
%-*-latex-*-
\input{myquizpreamble}
\input{yliow}
\input{ciss245}
\textwidth=6in

\renewcommand\TITLE{Final f01}

\renewcommand\EMAIL{}
\newcommand\POINTS{\textwhite{xxx/xxx}}


\renewcommand\AUTHOR{bglandis1@cougars.ccis.edu} % CHANGE TO YOURS

\begin{document}
\topmattertwo

When submitting using alex, enter \verb!f01! for work.

%-*-latex-*-
\textsc{Instructions}
\begin{enumerate}
\li This is a closed-book, no-discussion, no-calculator, no-browsing-on-the-web
    no-compiler/no-MIPS-simulator test.
\li Cheating is a serious academic offense. If caught you will 
    receive an immediate score of -100\%.
\li If a question asks for a program output and the program or
    code fragment contains
    an error, write \verb!ERROR! as output.
    When writing output, whitespace is significant.
\li If a question asks for the computation of a value and the program or
    code fragment contains
    an error, write \verb!ERROR! as value.
\li When you're asked to write a C++ statement, don't forget that it must
    end with a semicolon.
\li Bubblesort refers to the bubblesort algorithm in our notes
    where values are sorted in ascending order.
\end{enumerate}

\vspace{1cm}


\begin{center}
  \textsc{Honor Statement}
\end{center}
I, \answerbox{BRYSEN LANDIS},
attest to the fact that the submitted work is my own and
is not the result of plagiarism.
Furthermore, I have not aided another student in the act of
plagiarism.

% ------------------------------------------------------------------------------
\begin{python}
from scoretable import *
\end{python}
        
%-------------------------------------------------------------------------------
\newpage
\nextq
What is the output of this code fragment
if the user enters \verb!0! for \verb!i!, \verb!1! for \verb!j!,
and \verb!2! for \verb!k!?
\begin{console}
int i;
int j;
int k;
std::cin >> i >> j >> k;
int * p = &i;
int * q = &j;
int * r = &k;
q = p;
p = r;
std::cout << *p << ' ' << *q << ' ' << *r << '\n'; 
\end{console}

\textsc{Answer:}
\begin{answercode}
2 0 2
\end{answercode}




%------------------------------------------------------------------------------
\newpage
\nextq
At one point in the execution of a program, you have the following memory model:
\begin{center}
\includegraphics[width=4in]{pic01.PNG}
\end{center}
Is it possible to execute one or more statements to get the following:
\begin{center}
\includegraphics[width=4in]{pic02.PNG}
\end{center}
If it is, write the statements (use the least number please).
If it isn't explain why.
\textsc{Answer:}
\begin{answercode}
int ** t = p;
p = q;
q = t;
\end{answercode}


%------------------------------------------------------------------------------
\newpage
\nextq
The following code fragment has a memory leak.
Correct the code to ensure there's no memory leak.
(Make corrections to the code below directly.)

\textsc{Answer:}
\begin{answercode}
int f(int n)
{
    int * a = new int[n];
    int * i = new int;

    a[0] = rand();

    for (*i = 1; *i < n; ++(*i))
    {
        a[*i] += rand();
        a[*i] -= a[*i - 1];
    }
    int ret = a[n - 1];

    delete[] a;
    delete i;

    return ret;
}
\end{answercode}

%------------------------------------------------------------------------------
\newpage
\nextq
The following works with a 2D array that is in the heap.
(Look at the double for-loop.)
Complete the declaration of \verb!p!, allocate memory appropriately,
and at the end of the code fragment, deallocate memory used.

\textsc{Answer:}
\begin{answercode}
// Declare p and allocate memory for p.

int ** p = new int*[10];
for (int i = 0; i < 10; ++i)
{
    p[i] = new int[20];
}
  
for (int r = 0; r < 10; ++r)
{
    for (int c = 0; c < 20; ++c)
    {
        p[r][c] = 42;
    }
}

// Deallocate memory use by p.

for (int i = 0; i < 10; ++i)
{
  delete[] p[i]; 
}

delete[] p;

\end{answercode}


%-------------------------------------------------------------------------------
\newpage
\nextq
The following program computes and prints the minimum of an array of 10 integer
values entered by the user.
Follow the instructions (in comments) in order to complete the program.
Do not remove the comments.
I have already declared all the variables you need.
Therefore do not declare any other variables.
You must allocate and deallocate memory correctly.

\textsc{Answer:}
\begin{answercode}
#include <iostream>

int main()
{
    int * p;
    int * q;
    int * min; 

    // Allocate memory for p so that p points to an array of 10 values
    // in the memory heap.

    p = new int[10];

    // Prompt user for 10 integers and put them in the array that p
    // points to.
    for (q = p; q < p + 10; ++q)
    {
      std::cin >> *q;
    }

    // Compute the minimum of values in the array of 10 values that p
    // points to and store in integer that min points to.
    min = p;
    for (q = p + 1; q < p + 10; ++q)
    {
        if (*q < *min)
        {
            min = q;
        }
    }

    std::cout << *min << std::endl;

    // Deallcate memory used by p
    delete[] p;
    return 0;
}
\end{answercode}


%-------------------------------------------------------------------------------
\newpage
\nextq
Write a function that checks if a string is a palindrome.
A palindrome is a string that is it’s own \lq\lq reverse".
In other words, you read the string left-to-right and right-to-left,
you get the same string. For instance
\[
	\texttt{"madam"}
\]
Another example is 
\[
	\texttt{"tacocat"}
\]

Write a function \verb!is_two_palindromes!
that checks if string is made up of two palindromes.
For instance
\[
\texttt{"madammom"}
\]
is made up of the concatenation of two palindromes: \texttt{madam} and
\texttt{mom}.

Don’t forget that a C-string is null-terminated, i.e.,
there’s an end-of-data character of \verb!'\0'! in the string that indicates
where the string data ends.
You \textit{cannot} use any C-string functions or the C\texttt{++} string class.
You can add as many functions as you like.
I've added an \verb!is_palindrome! function you can choose to use that or not.

\textsc{Answer:}
\begin{answercode}
bool is_palindrome(char * start, char * end)
{
  end--;
  while (start < end)
  {
    if(*start != *end) return false;
    start++;
    end--;
  }

  return true;

}

bool is_two_palindromes(char s[])
{
  char * end = s;
  while (*end != '\0') end++;

  for (char * mid = s + 1; mid < end; mid++)
  {
    if (is_palindrome(s, mid) && is_palindrome(mid, end))
    {
      return true;
    }
  }
  return false;
}
\end{answercode}


%-------------------------------------------------------------------------------
\newpage
\nextq
Complete the following \verb!Rectangle! class given the following requirements:
\begin{enumerate}[nosep]

\li There are two private member variables \verb!width_! and \verb!height_!,
    both of \verb!double! type.

\li You can construct a \verb!Rectangle! object of a given width (say 10) and
    given height (say 5) as follows:
    \begin{Verbatim}[frame=single,fontsize=\small]
Rectangle r(10, 5);
    \end{Verbatim}
You must use an initializer list whenever possible.

\li You can also construct a \verb!Rectangle! object (say \verb!r1!) with
    another \verb!Rectangle! object (say \verb!r2!) so that
    \verb!r1! and \verb!r2! have the same width and height:
    \begin{Verbatim}[frame=single,fontsize=\small]
Rectangle r1(r2);
    \end{Verbatim}
    You must use an initializer list whenever possible.

\li You can get the value of the \verb!width_! of a Rectangle object
    (say \verb!r!) as follows:
    \begin{Verbatim}[frame=single,fontsize=\small]
std::cout << r.get_width() << std::endl;
    \end{Verbatim}

\li You can also get the value of the \verb!width_! of a
    \verb!Rectangle! object (say \verb!r!) as follows:
    \begin{Verbatim}[frame=single,fontsize=\small]
std::cout << r.width() << std::endl;
    \end{Verbatim}

\li You can set the value of the \verb!width_! of a \verb!Rectangle! object
    (say \verb!r!) as follows:
    \begin{Verbatim}[frame=single,fontsize=\small]
r.set_width(12.34);
    \end{Verbatim}

\li You can also set the value of the \verb!width_! of a \verb!Rectangle!
    object (say \verb!r!) as follows:
    \begin{Verbatim}[frame=single,fontsize=\small]
r.width() = 12.34;
    \end{Verbatim}
    
\li You can also get the area of a \verb!Rectangle! object (say \verb!r!) as
    follows:
    \begin{Verbatim}[frame=single,fontsize=\small]
std::cout << r.area() << std::endl;
    \end{Verbatim}
    
\end{enumerate}
All methods are to be implemented in the header file below.

(Turn page)

\newpage
\textsc{Answer:}
\begin{answercode}
// file: Rectangle.h

#ifndef RECTANGLE_H
#define RECTANGLE_H

class Rectangle
{
  public:
  Rectangle(double w, double h) 
  : width_(w), height_(h)
  {}
  
  Rectangle(const Rectangle & r)
  : width_(r.width_), height_(r.height_)
  {}
  
  double get_width() const
  {
    return width_;
  }
  
  double width() const
  {
    return width_;
  }
  
  double & width()
  {
    return width_;
  }
  
  void set_width(double w)
  {
    width_ = w;
  }
  
  double area() const
  {
    return width_ * height_;
  }

  private:
  double width_;
  double height_;
};

#endif
\end{answercode}


%-------------------------------------------------------------------------------
\newpage
\nextq
It has been decreed that every living thing in the Milky Way galaxy must have
an integer \verb!id! (an integer).
A human being of course is a living thing,
but a human being has a gender (a character) and number of heads, arms, and
legs.
Create three classes: \verb!LivingThing!, \verb!Earthling! and
\verb!Martian! where
\verb!Earthling! is a subclass of \verb!LivingThing!.
You only need to have enough code for the following code to execute:
\begin{console}[fontsize=\small]
// john has id 3423452, is male, has 2 heads, 3 arms, and 4 legs
Earthling john(3423452, 'm', 2, 3, 4);
\end{console}
A martian also has an id, but no gender, have heads and legs, but no arms.
All member variables must be private.
Furthermore, based on the information above, various private member variables
must be placed in the right class to minimize code duplication

Implement the \verb!LivingThing! class below.
(See next questions for the \verb!Earthling! and \verb!Martian! class.)

\textsc{Answer:}
\begin{answercode}
// file: LivingThing.h

#ifndef LIVINGTHING_H
#define LIVINGTHING_H


class LivingThing
{
  public:
  LivingThing(int id) : id_(id) {}
  
  private:
  int id_;
};

#endif
\end{answercode}


%-------------------------------------------------------------------------------
\newpage
\nextq
This is a continuation of the previous question.
Implement the \verb!Earthling! class below.

\textsc{Answer:}
\begin{answercode}
// file: Earthling.h
#ifndef EARTHLING_H
#define EARTHLING_H

#include "LivingThing.h"

class Earthling : public LivingThing
{
  public:
  Earthling(int id, char gender, int heads, int arms, int legs)
  : LivingThing(id), gender_(gender), heads_(heads), arms_(arms), legs_(legs)
  {}
  
  private:
  char gender_;
  int heads_;
  int arms_;
  int legs_;
};

#endif

\end{answercode}

%-------------------------------------------------------------------------------
\newpage
\nextq
This is a continuation of the previous question.
Implement the \verb!Martian! class below.

\textsc{Answer:}
\begin{answercode}
// file: Martian.h
#ifndef MARTIAN_H
#define MARTIAN_H

#include "LivingThing.h"

class Martian : public LivingThing
{
  public:
  Martian(int id, int heads, int legs)
  : LivingThing(id), heads_(heads), legs_(legs)
  {}
  
  private:
  int heads_;
  int legs_;
};

#endif


\end{answercode}

%-------------------------------------------------------------------------------
\newpage
\nextq
The following class allows us to create objects that models rational numbers:
\begin{console}
class Rational
{
public:
    Rational(int n, int d)
        : n_(n), d_(d)
    {}
    // other methods not shown
private:
    int n_;
    int d_;
};
\end{console}
Complete two methods that allows you to perform additions
(\verb!+! and \verb!+=!).
As an example, the following usage of the Rational class should work correctly:
\begin{console}
Rational r0(1, 2), r1(3, 4); // r0 models 1/2 and r1 models 3/4
Rational r2 = r0 + r1;       // r2 models 1/2 + 3/4 , 
                             // i.e., (1 * 3 + 2 * 4) / (2 * 4)
Rational r3 = r2;
r3 += r2;                    // r3 models r3 + r2
\end{console}
Your code \textit{must} be minimal.
(HINT: \verb!+! should use \verb!+=!.)

(Turn page)

\newpage
\textsc{Answer:}
\begin{answercode}
// file: Rational.h

#ifndef RATIONAL_H
#define RATIONAL_H

class Rational
{
public:
    Rational(int n, int d)
        : n_(n), d_(d)
    {}

    Rational & operator+=(const Rational & r)
    {
      n_ = n_ * r.d_ + d_ * r.n_;
      d_ = d_ * r.d_;
      return *this;
    }
    
    Rational operator+(const Rational & r) const
    {
      Rational result = *this;
      result += r;
      return result;
    }
    
private:
    int n_;
    int d_;
};

#endif
\end{answercode}


%-------------------------------------------------------------------------------
\newpage
\nextq
The following class allows us to create objects that handle dynamic array of
integers, i.e., each object contains a pointer to an array in the heap:
\begin{console}[fontsize=\small]
class IntDynArr
{
public:
    IntDynArr(int size)
        : p_(new int[size]), size_(size), capacity_(size)
    {}
    // other methods not shown
private:
    int * p_;
    int size_;
    int capacity_;
};
\end{console}
Of course since objects of the class uses a resource that is not automatically
released, you have to write the
\begin{enumerate}[nosep]
\li Destructor
\li Copy constructor
\li Assignment operator
\end{enumerate}
in order to overwrite the default ones:
\begin{console}[fontsize=\small]
class IntDynArr
{
public:
    IntDynArr(int size)
        : p_(new int[size]), size_(size), capacity_(size)
    {}

    ~IntDynArr();
    IntDynArr(const IntDynArr &);
    IntDynArr & operator=(const IntDynArr &);

    // other methods not shown
private:
    int * p_;
    int size_;
    int capacity_;
};
\end{console}

Implement the destructor below.
(The copy constructor and assignment operator are in the next two
questions.)

\textsc{Answer:}
\begin{answercode}
// file: IntDynArray.cpp

#include "IntDynArray.h"

IntDynArray::~IntDynArr()
{
  delete[] p_;
}

// other methods not shown
\end{answercode}


%-------------------------------------------------------------------------------
\newpage
\nextq
This is a continuation of the previous question.
Implement the copy constructor below for \verb!IntDynArray!.
(Constructors not using initializer lists as much as possible are considered
incorrect.)

\textsc{Answer:}
\begin{answercode}
// file: IntDynArray.cpp

#include "IntDynArray.h"

IntDynArray::IntDynArr(const IntDynArr & x)
: p_(new int[x.capacity_]), size_(x.size_), capacity_(x.capacity_)
{
  for (int i = 0; i < size_; ++i)
  {
    p_[i] = x.p_[i];
  }
}

// other methods not shown
\end{answercode}


%-------------------------------------------------------------------------------
\newpage
\nextq
This is a continuation of the previous question.
Implement the assignment operator below for \verb!IntDynArray!.

\textsc{Answer:}
\begin{answercode}
// file: IntDynArray.cpp

#include "IntDynArray.h"

IntDynArray & IntDynArray::operator=(const IntDynArr & x)
{
  if (*this != &x)
  {
    delete[] p_;
    p_ = new int[x.capacity_];
    size_ = x.size_;
    capacity_ = x.capacity_;
    for (int i = 0; i < size_; ++i)
    {
      p_[i] = x.p_[i];
    }
  }
  return *this;
}

// other methods not shown
\end{answercode}


%-------------------------------------------------------------------------------
\newpage
\nextq
Complete the following recursive function \verb!sumsquares()! so that
\verb!sumsquares(n)! computes $0^2 + 1^2 + 2^2 + \cdots + n^2$, i.e.,
it computes the \lq\lq sum of squares".
$n$ is an integer that is at least $0$.

\textsc{Answer:}
\begin{answercode}
int sumsquares(int n)
{
  if (n = 0) return 0;
  return n * n + sumsquares(n - 1);
}
\end{answercode}

%-------------------------------------------------------------------------------
\newpage
\nextq
Write a recursive function to print the integers from address
\verb!start! up to \textit{but not including} address \verb!end!.
Between every two integers, the function prints a space.
After all integers are print, the function prints a newline character.

\textsc{Answer:}
\begin{answercode}
void println(int * start, int * end)
{
  if (start >= end)
  {
    std::cout << '\n';
    return;
  }
  std::cout << *start;
  if (start + 1 < end)
  {
    std::cout << ' ';
  }
  println(start + 1, end);
}
\end{answercode}

%-------------------------------------------------------------------------------
\newpage
\nextq
Complete the following function that performs bubblesort on
\verb!*start!, \verb!*(start+1)!, ..., \verb!*(end-1)!
where \verb!x! is an array of integer values.
Both functions must be recursive.

\textsc{Answer:}
\begin{answercode}
// Performs one pass of bubblesort on *start, *(start + 1), ...,
// *(end - 1).
void bubblesort_pass(int * start, int * end)
{
  if (start + 1 >= end) return;
  if (*start > *(start + 1))
  {
    int t = *start;
    *start = *(start + 1);
    *(start + 1) = t;
  }
  bubblesort_pass(start + 1, end);
}

// Performs bubblesort on *start, *(start + 1), ..., *(end - 1).
// Uses bubblesort_pass.
void bubblesort(int * start, int * end)
{
  if (start >= end) return;
  bubblesort_pass(start, end);
  bubblesort(start, end - 1); 
}
\end{answercode}

%-------------------------------------------------------------------------------
\newpage
\nextq
Convert the following \verb!vec2d!, which models a vector in 2D space with
two \verb!doubles! to a class template, \verb!vec2!,
which models a vector in 2D space with two \verb!T! values.
You only need to modify what's in the given class.
Do not implement any extra methods/functions not shown below.
Furthermore, if the \verb!operator[]! is called with a value that is not
\verb!0! or \verb!1!, an \verb!IndexError! object is thrown as an exception.

\textsc{Answer:}
\begin{answercode}
// file: vec2d.h

#ifndef VEC2D_H
#define VEC2D_H

class IndexError
{};

template <typename T>
class vec2
{
  public:
  vec2(T x, T y)
  : x_(x), y_(y)
  {}
  T operator[](int i) const
  {
    if (i != 0 && i != 1) throw IndexError();
    return (i == 0 ? x_ : y_);
  }
  T & operator[](int i)
  {
    if (i != 0 && i != 1) throw IndexError();
    return (i == 0 ? x_ : y_);
  }
  private:
  T x_;
  T y_;
};

#endif
\end{answercode}


%-------------------------------------------------------------------------------
\newpage
\nextq
The class below allows me to do this:
\begin{console}[fontsize=\small]
Square square;
std::cout << square.compute(5) << '\n'; // prints 25
\end{console}
Modify the only method in the class so that I get a functor, i.e.,
so that I can do this instead:
\begin{console}[fontsize=\small]
Square square;
std::cout << square(5) << '\n'; // prints 25
\end{console}

\textsc{Answer:}
\begin{answercode}
// file: square.h

#ifndef SQUARE_H
#define SQUARE_H

class Square
{
  int operator()(int x)
  {
    return x * x;
  }
};

#endif
\end{answercode}

%-------------------------------------------------------------------------------
\newpage
\nextq
The following will print \verb!"A::f()\n"! twice.
Modify it (add one C\texttt{++} keyword) so that one of the outputs is
\verb!"A::f()\n"! and the other is \verb!"B::f()\n"!. 

\textsc{Answer:}
\begin{answercode}
#include <iostream>

class A
{
public:
    virtual void f() const
    {
        std::cout << "A::f()\n";
    }
};

class B: public A
{
public:
    virtual void f() const
    {
        std::cout << "B::f()\n";
    }
};

int main()
{
    B b1;
    A a1 = b1; 
    a1.f();

    B b2;
    A * a2 = &b2;
    a2->f();

    return 0;
}
\end{answercode}

%-------------------------------------------------------------------------------
\newpage
\nextq
The following code (obviously!) does not work.
Correct it so that the output is
\begin{Verbatim}[frame=single,fontsize=\footnotesize]
grrr ... i'm a wumpus
hi ... i'm a hunter
\end{Verbatim}

\textsc{Answer:}
\begin{answercode}
#include <iostream>
#include <vector>

class LivingThing
{
public:
  virtual void print() const = 0;
};

class Wumpus : public LivingThing
{
public:
    virtual void print() const
    {
        std::cout << "grrr ... i'm a wumpus\n";
    }
};

class Hunter : public LivingThing
{
public:
    virtual void print() const
    {
        std::cout << "hi ... i'm a hunter\n";
    }
};

int main()
{
    std::vector< LivingThing > things;
    things.push_back(Wumpus());
    things.push_back(Hunter());

    for (int i = 0; i < 2; ++i)
    {
        things[i].print();
    }
    
    return 0;

}
\end{answercode}

%-------------------------------------------------------------------------------
\newpage
\nextq
Complete the following that prints the values of \verb!x! in ascending order
\textit{without} changing the values of \verb!x!.
In other words
the value of \verb!x[0]! stays the same,
the value of \verb!x[1]! stays the same, etc.
The array in \verb!main()! is only a test case.
Your program must work for any integer array.

\begin{answercode}
#include <iostream>

void bubblesort(int ** start, int ** end)
{
  for (int ** i = start; i < end - 1; ++i)
  {
    for (int ** j = start; j < end - 1; ++j)
    {
      if **j > **(j + 1))
      {
        int * t = *j;
        *j = (*j + 1);
        (*j + 1) = t;
      }
    }
  }
}

int main()
{
    const int n = 5;            
    int x[n] = {5, 3, 1, 2, 4};
    
    int * px[n];
    for (int i = 0; i < n; ++i)
    {
        px[i] = &x[i];
    }

    bubblesort(&px[0], &px[n]);
    
    // The original array x is unchanged.
    for (int *p = &x[0]; p < &x[n]; ++p)
    {
        std::cout << (*p) << ' ';
    }
    std::cout << '\n';

    // Prints the values in x in ascending order.
    for (int ** p = &px[0]; p < &px[n]; ++p)
    {
        std::cout << (**p) << ' ';
    }
    std::cout << '\n';
    
    return 0;
}
\end{answercode}

%------------------------------------------------------------------------------
\newpage
\input{instructions}
\end{document}
