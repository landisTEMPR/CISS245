%-*-latex-*-
%-*-latex-*-
\input{myquizpreamble}
\input{yliow}
\input{ciss245}
\textwidth=6in

\renewcommand\TITLE{Final f01}

\renewcommand\EMAIL{}
\newcommand\POINTS{\textwhite{xxx/xxx}}


\renewcommand\AUTHOR{} % CHANGE TO YOURS
  
\begin{document}
\topmattertwo

\newcommand\abox[1]{
  \begin{flushright}
    ANSWER:
    \newsolutionboxone{0.5in}{0.25in}{\instructoronly{#1}}
  \end{flushright}
  }

\begin{python}
from latextool_basic import *
honor_statement()
\end{python}

%------------------------------------------------------------------------------
\newpage
\begin{python}
from latextool_basic import *
test_score_table()
\end{python}

This is a
\begin{enumerate}
  \li Closed-book
  \li Closed-notes
  \li No calculator
  \li No computer (except your brain)
  \li No discussion (except with yourself)
\end{enumerate}
Write legibly.

%==============================================================================
\newpage
\textsc{How to describe a binary tree in text}

For the above questions,
here's an example on how to describe a binary tree in text.
For this tree
\begin{console}[fontsize=\footnotesize]
            (10)
          /      \
      ( 4)        (25)
     /           /    \
 ( 1)        (12)      (27)
                           \
                            (29)          
\end{console}
you write
\begin{console}[fontsize=\footnotesize]
[10, [4, 25]]
[4, [1, None]]
[25, [12, 27]]
[1, [None, None]]
[12, [None, None]]
[27, [None, 29]]
[29, [None, None]]
\end{console}

%==============================================================================
\newpage
Q1. [Derive big--O of runtime of a given algorithm]

The following computes the sum of squares from $1^2$ to $n^2$:
\begin{Verbatim}[frame=single]
s = 0
for i = 0, ..., n - 1:
    term = i * i
    s = s * s + term
\end{Verbatim}
Here's the program with goto statements and timing for each statement:
\begin{Verbatim}[frame=single]
         i = 1              time t0
         s = 0              time t1
LOOP:    if i > n:          time t2
             goto ENDLOOP   time t3
         term = i * i       time t4
         s = s + term       time t5
         i = i + 1          time t6
         goto LOOP          time t7
ENDLOOP:
\end{Verbatim}
(a) Compute the time taken as a function of $n$.
The expression should contain constants $t_0$, ..., $t_7$.
\\
\ANSWER
\begin{answerlong}
$T(n) = ?$
\end{answerlong}

(b) Simplify the runtime function by giving names $A$, $B$, ...
to the constants of the function from (a).
\\
\ANSWER
\begin{answerlong}
$T(n) = ?$
\end{answerlong}

(c) Fudge away the constants and write down the simplest $g(n)$ such 
that the time in (b) is a big-O of your $g(n)$.
Your $g(n)$ should be either $n^0$ (i.e., 1) or $n$ or $n^2$ or $n^3$ or ...
\\
\ANSWER
\begin{answerlong}
$T(n) = ?$
\end{answerlong}

%==============================================================================
\newpage
Q2. [Recognize big--O of algorithm]

(a)
What is the big--O of the runtime of the following algorithm:
\begin{Verbatim}[frame=single]
s = 0
for i = 0, 1, 2, ..., n - 1:
    for j = 0, 1, 2, ..., i:
        for k = j, j + 1, ..., n - 1:
            s = s + x[i][j] + k
\end{Verbatim}
\ANSWER
\begin{answerlong}
$T(n) = ?$
\end{answerlong}

(b)
What is the big--O of the runtime of the following algorithm:
\begin{Verbatim}[frame=single]
s = 0
for i = 0, 1, 2, ..., n - 1:
    for j = 0, 1, 2, 3::
        for k = i, i + 1, ..., n - 1:
            s = s + x[i][k] * j
\end{Verbatim}
\ANSWER
\begin{answerlong}
$T(n) = ?$
\end{answerlong}

(c)
What is the big--O of the best runtime of the following algorithm:
\begin{Verbatim}[frame=single]
s = 0
for i = 0, 1, 2, ..., n - 1:
    if x[0][0] > 0:
        for j = 0, 1, 2, 3::
            for k = i, i + 1, ..., n - 1:
                s = s + x[i][k] * j
\end{Verbatim}
\ANSWER
\begin{answerlong}
$T(n) = ?$
\end{answerlong}

%==============================================================================
\newpage
Q3. [Recall big-O runtime performance of standard algorithms]

For questions on runtime,
\underline{just state} the big-O of the runtime performance.


(a) What is the worst runtime of mergesort?
\\
\ANSWER
\begin{answerlong}
$T(n) = ?$
\end{answerlong}
  
(b) What is the best runtime of mergesort?
\\
\ANSWER
\begin{answerlong}
$T(n) = ?$
\end{answerlong}
  
(c) What is the average runtime of mergesort?
\\
\ANSWER
\begin{answerlong}
$T(n) = ?$
\end{answerlong}
  
(d) What is the worst runtime of quicksort?
\\
\ANSWER
\begin{answerlong}
$T(n) = ?$
\end{answerlong}

(e) What is the best runtime of quicksort?
\\
\ANSWER
\begin{answerlong}
$T(n) = ?$
\end{answerlong}

(f) What is the average runtime of quicksort?
\\
\ANSWER
\begin{answerlong}
$T(n) = ?$
\end{answerlong}


(g) What is the worse runtime of heapsort?
\\
\ANSWER
\begin{answerlong}
$T(n) = ?$
\end{answerlong}

%==============================================================================
\newpage
Q4. [Trace of basic sorting algorithms]

Perform selection sort on the following array, showing
the result after each pass:
\[
\texttt{ \{5, 1, 2, 4, 3, 6\} } 
\]
\ANSWER
\begin{answercode}
{5, 1, 2, 4, 3, 6}

\end{answercode}


%==============================================================================
\newpage
Q5.

(a) 
What are the correct intermediate steps of the following
\[
15,20,10,18
\]
when it is
being sorted with the quicksort?
Write down \textit{all} that applies.
(Each stage is the array after pivoting and partitioning.
There are of course many methods for choosing a pivot 
and many ways to partition -- 
you do not need to know the specific method to answer this question.)
\begin{enumerate}
\item[(A)] 15,10,20,18 $\rightarrow$ 15,10,18,20 $\rightarrow$ 10,15,18,20
\item[(B)] 15,20,10,18 $\rightarrow$ 15,10,20,18 $\rightarrow$ 10,15,20,18
\item[(C)] 10,20,15,18 $\rightarrow$ 10,15,20,18 $\rightarrow$ 10,15,18,20
\item[(D)] 10,20,15,18 $\rightarrow$ 10,18,15,20 $\rightarrow$ 10,15,18,20
\item[(E)] None of the above.
\end{enumerate}
\ANSWER
\begin{answercode}

\end{answercode}


(b) Write down \textit{all} that applies: Quicksort uses
\begin{enumerate}
\item[(A)] finding maximum 
\item[(B)] partitioning
\item[(C)] finding a pivot
\item[(D)] merging
\end{enumerate}
\ANSWER
\begin{answercode}

\end{answercode}


(c) Write down one of the two: Quicksort is a stable sort.
\begin{enumerate}
  \item[(A)] True
  \item[(B)] False
\end{enumerate}
\ANSWER
\begin{answercode}

\end{answercode}


%==============================================================================
\newpage
Q6.
You are given the following singly linked node class:
\begin{Verbatim}[frame=single,fontsize=\footnotesize]
class Node
{
public:
    Node(int key=0, Node * next=NULL)
        : key_(key), next_(next)
    {}
    
    int key_;
    Node * next_;
};
\end{Verbatim}
Note that all members are public (to make life simple).
The following is a class for singly linked list (of integers):
\begin{Verbatim}[frame=single,fontsize=\footnotesize]
class SLList
{
public:
    SLList()
        : phead_(NULL)
    {}
    
    // other methods
    
private:
    Node * phead_;
};
\end{Verbatim}

Complete the following methods that removes the head node
if the linked list is not empty.
If the linked list is empty (i.e., \verb!phead_! is \verb!NULL!),
nothing is done (i.e., do not throw any exception).
Note that the return type is \verb!void!, i.e., do not
return any value.
\ANSWER
\begin{answercode}
void SLList::delete_head()
{








  
}
\end{answercode}




%==============================================================================
\newpage
Q7. [Doubly linked list]

You are given an incomplete skeleton of a doubly linked list implementation:
\begin{Verbatim}[frame=single,fontsize=\footnotesize]
class Node
{
public:
    Node(int key=0, Node * prev=NULL, Node * next=NULL)
        : key_(key), prev_(prev), next_(next)
    {}
    int key_;
    Node * prev_;
    Node * next_;
};

class DLList
{
public:
    DLList()
        : pheadsentinel_(new Node(0, NULL, NULL)),
          ptailsentinel_(new Node(0, NULL, NULL))
    {
        pheadsentinel_->next_ = ptailsentinel;        
        ptailsentinel_->prev_ = pheadsentinel;        
    }
private:
    Node * pheadsentinel_;  
    Node * ptailsentinel_;  
};
\end{Verbatim}

Complete the following method to perform insert tail.

\ANSWER
\begin{answercode}
void DLList::insert_tail(int key)
{



  
}
\end{answercode}




%==============================================================================
\newpage
Q8. [Tree]
\begin{python}
from latextool_basic import *
print(graph(shape='circle',
        yscale=1.3,
        minimum_size='5mm',
        layout='''
             a
         b   c   d
        e   f g h i j
       k l     m  
        ''',
        edges='a>b,a>c,a>d,b>e,c>f,c>g,d>h,d>i,d>j,e>k,e>l,h>m',
        ))
\end{python}

(a) List the values of the nodes using pre-order traversal:

\ANSWER
\begin{answercode}

\end{answercode}


(b) List the values of the nodes using post-order traversal:

\ANSWER
\begin{answercode}

\end{answercode}


(c) What is the height of the tree?

\ANSWER
\begin{answercode}

\end{answercode}


(d) What is the depth of $h$?

\ANSWER
\begin{answercode}

\end{answercode}


(e) What are the siblings of $i$?

\ANSWER
\begin{answercode}

\end{answercode}


(f) Suppose \verb!Node! is a binary tree node class.
Assume that the class has \verb!get_left()! and \verb!get_right()!
methods to return the left and right pointer of a \verb!Node!
object.
Complete the following function that computes the height at \verb!p!.
You may assume that there's a \verb!max()! function such that
\verb!max(a, b)! returns the maximum of \verb!a! and \verb!b!.

\ANSWER
\begin{answercode}
int height(const Node * const p)
{
  
}
\end{answercode}




%==============================================================================
\newpage
Q9. [BST]

Write down the runtimes of the following operations on a BST
(binary search tree) that has $n$ nodes.

(a) Best runtime for insert (when given key):
\\
\ANSWER
\begin{answerlong}

\end{answerlong}

(b) Worst runtime for insert (when given key):
\\
\ANSWER
\begin{answerlong}

\end{answerlong}

(c) Best runtime for delete (when given key): 
\\
\ANSWER
\begin{answerlong}

\end{answerlong}

(d) Worst runtime for delete (when given key):
\\
\ANSWER
\begin{answerlong}

\end{answerlong}

(e) Worst runtime for search (returning pointer when given key):
\\
\ANSWER
\begin{answerlong}

\end{answerlong}

(f) You are given this BST:
\begin{python}
from latextool_basic import *
print(r'''
\begin{center}
%s
\end{center}
''' % graph(shape='circle',
        yscale=1.3,
        minimum_size='5mm',
        layout='''
           a
         b   c  
        e d f g 
       k l n p 
        ''',
        edges='a>b,a>c,b>d,b>e,c>f,c>g,e>k,e>l,f>n,f>p',
        a='label=40',
        b='label=20',
        c='label=60',
        d='label=25',
        e='label=5',
        f='label=50',
        g='label=70',
        k='label=1',
        l='label=10',
        n='label=45',
        p='label=55',
        ))
\end{python}
Describe the resulting BST in text after deleting the value 20.
(Recall our choice of \lq\lq always using left subtree whenever possible''
when doing delete.)
\\
\ANSWER
\begin{answercode}

\end{answercode}




%==============================================================================
\newpage
Q10. [Balancing tree]
Let $T$ be the following tree:
\begin{python}
from latextool_basic import *
print(r'''
\begin{center}
%s
\end{center}
''' % graph(shape='circle',
        yscale=1.3,
        minimum_size='5mm',
        layout='''
           a
         b   c  
        e d f g 
       k l n p 
        ''',
        edges='a>b,a>c,b>d,b>e,c>f,c>g,e>k,e>l,f>n,f>p',
        ))
\end{python}

(a) Describe the tree $T$ in text after a right rotation at $c$.
\\
\ANSWER
\begin{answercode}

\end{answercode}

(b) Draw the tree $T$ after a left rotation at  $a$.
(Note: $T$ refers the tree at the beginning
of the question and not the resulting tree of (a).)
\\
\ANSWER
\begin{answercode}

\end{answercode}

(c) Going back to the original tree $T$ in (a), suppose
$q$ was inserted on the left of $k$ (and the tree was re-balanced if necessary),
followed by
inserting $r$ on the right of $q$ (and the tree was re-balanced if necessary),
followed by
inserting $s$ on the \textred{\textso{left of $r$} right of $q$}
(and the tree was re-balanced if necessary),
Describe the resulting tree in text.
\\
\ANSWER
\begin{answercode}









\end{answercode}

(d) Going back to the original tree $T$ in (a), suppose
$q$ was inserted on the right of $k$ (and the tree was re-balanced if necessary)
followed by inserting $r$ on the left of $q$ (and the tree was re-balanced if
necessary).
Describe the resulting tree in text.
\\
\ANSWER
\begin{answercode}

\end{answercode}




%==============================================================================
\newpage
Q11. [Heaps]

Starting with an empty max heap (as an array of course),
perform the following operations.
Show the result of each step after each heapify operation.
Describe the heap in text like an array such as \verb!{1, 2, 3, 4, 5}!
Note that the heap is a \textit{max} heap.


(a) Insert 5:
\\
\ANSWER
\begin{answercode}
{5}
\end{answercode}


(a) Insert 1:
\\
\ANSWER
\begin{answercode}
{}
\end{answercode}

(b) Insert 2:
\\
\ANSWER
\begin{answercode}
{}
\end{answercode}

(c) Insert 4:
\\
\ANSWER
\begin{answercode}
{}
\end{answercode}

(d) Insert 3:
\\
\ANSWER
\begin{answercode}
{}
\end{answercode}

(e) Insert 6:
\\
\ANSWER
\begin{answercode}
{}
\end{answercode}

(f) Delete max (the root):
\\
\ANSWER
\begin{answercode}
{}
\end{answercode}

(g) Delete max (the root):
\\
\ANSWER
\begin{answercode}
{}
\end{answercode}




%==============================================================================
\newpage
Q12. [Heapsort]

(a)
Consider the following array:
\[
\texttt{ \{5, 1, 0, 2, 3, 7, 3, 4\} }
\]
Perform operations on the array to build a maxheap.
Draw the array after \textit{every} swap.

\ANSWER
\begin{answercode}
{5, 1, 0, 2, 3, 7, 3, 4}

\end{answercode}


\newpage
(b) You are given a maxheap (as an array). Perform heapsort on the array,
drawing the array after \textit{every} swap.
\[
\texttt{ \{7,5,6,3,2,4,0,1\} }
\]

\ANSWER
\begin{answercode}
{7, 5, 6, 3, 2, 4, 0, 1}

\end{answercode}





%==============================================================================
\newpage
Q13. [Hash table]

Let the following function
\[
h(x) = 2 x + 1
\]
be used as a hash function.

(a)
In a hash table of size 8, insert the keys
1,3,2,5,4,8,6,7.
You must insert the keys in the given order, going left to right.
(Of course it the hash value is too huge, you mod is by the size of the table.)
Draw the table below.
The first number for each row is the index of that row,
the second is for a flag that indicates the state of the row:
\begin{tightlist}
  \li If a row is not occupied, the flag value is Available.
  \li If a row is occupied, the flag value is Not-Available.
  \li If a row was previously occupied (but that row was then deleted),
  the flag value is Deleted.
\end{tightlist}
The third column is for the key.
The fourth column is for the value of a row and can be left blank.

As an example, if the index 0 row is occupied by key 99, then
the row at index 0 is
\[
\texttt{0:\ \{Not-Available, 99,\}}
\]
See table below for this example.

In the event of a hash collision, the collision resolution method you should
use is linear probe, i.e., the \lq\lq next row'' method.

\ANSWER
\begin{answercode}
0: {Not-Available, 99, }
1: { , , }
2: { , , }
3: { , , }
4: { , , }
5: { , , }
6: { , , }
7: { , , }
\end{answercode}
(The data at index 0 is only an example. Correct it.)

(b) Continue with (a).
For each of the following keys, write down the number of rows
accessed in order to find that key.
Write down the average number of rows accessed.
\\
\ANSWER
\begin{answercode}
key 1: 
key 2:
key 3:
key 4:
key 5:
key 6:
key 7:
key 8:
Average:
\end{answercode}
  
(c) Continue with (b).
What is the worst case in terms of number of rows accessed
during a find of a key \textit{not} in the table?
\\
\ANSWER
\begin{answercode}

\end{answercode}





%==============================================================================
\newpage
Q14. [Applications of linked list]

(a) Convert the following infix arithmetic expression
\[
  2 * 1 - 3 / (6 + 5 - 4) + 7 
\]
into RPN.
(Do \textit{not} simplify, i.e., all the above digits must appear
in the RPN expression.)
\\
\ANSWER
\begin{answercode}

\end{answercode}


(b) Evaluate the following arithmetic expression written in RPN
\begin{Verbatim}
                          3 4 + 7 2 1 + / - 6 5 * -
\end{Verbatim}
\ANSWER
\begin{answercode}

\end{answercode}

(c)
In part (b),
what is the maximum size of the stack during the computation?

\ANSWER
\begin{answercode}

\end{answercode}





%==============================================================================
\newpage
Q15.
(a) Describe an algorithm that you would
use to sort an array \verb!x! of 1000000000 integers with values
    in the following ranges: each \verb!x[i]! is in -100,...,-91
    or in 1000000, ..., 1000009.
    
\ANSWER
\begin{answercode}

\end{answercode}

(b)
Next, provide a C++ implementation.
Of course assume you want the fastest algorithm.
Your answer is incorrect if your runtime is not the best possible.

\ANSWER
\begin{answercode}
void sort(int x[])
{
    int t[           ]; // If your sorting is in-place, delete this line

      
}      
\end{answercode}




%==============================================================================
\newpage
Q16.
Implement mergesort to sort an array of integer values in ascending order.
The prototype is \verb!mergesort(int * start, int * end, int * t)!.
The array to be sorted is \verb!*start!, ..., \verb!*(end - 1)!.
You may assume \verb!t! points to a temporary array that is large
enough for your mergesort.
(You probably want a merge function.)

\ANSWER
\begin{answercode}

\end{answercode}





%==============================================================================
\newpage
Q17.
Suppose you are given the following node class for binary trees:
\begin{console}
class Node
{
public:
    int key_;
    Node * parent_;
    Node * left_;
    Node * right_;
};
\end{console}
Write a function \verb!deallocate! such that if \verb!proot!
points to the root of a binary tree (with nodes in the heap),
\verb!deallocate(proot)! will deallocate the memory used by this tree
and set \verb!proot! to \verb!NULL!.

\ANSWER
\begin{answercode}

\end{answercode}




%==============================================================================
\newpage
Q18.
Describe an algorithm $f$ such that $f(x, n, k)$ would
compute the $k$--th smallest value in an array $x[0..n - 1]$ of doubles.
For instance if the array is
\[
x = \{1.1, 5.5, 2.2, 7.7, 3.3, 4.4\}
\]
then $f(x, n, 3)$ would return the third smallest value of $x$, which is $3.3$.
Note that you want the fastest algorithm.

(Hint: Modify quicksort.)

\ANSWER
\begin{answercode}

\end{answercode}



\end{document}
